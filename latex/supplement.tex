\documentclass[12pt,letterpaper]{article}

\usepackage{amsmath}
\usepackage{fontspec}
% \setromanfont[Ligatures=TeX]{texgyrepagella-regular.otf}
% \usepackage{unicode-math}
% \setmathfont{texgyrepagella-math.otf}

\usepackage[margin=1.0in]{geometry}
\usepackage[figuresleft]{rotating}
\usepackage[longnamesfirst]{natbib}
\usepackage{dcolumn}
\usepackage{booktabs}
\usepackage{multirow}
\usepackage[flushleft]{threeparttable}
\usepackage{setspace}
\usepackage[justification=centering]{caption}
\usepackage[]{subcaption}
\usepackage[xetex,colorlinks=true,linkcolor=black,citecolor=black,urlcolor=black]{hyperref}
\usepackage{adjustbox}
\usepackage{nicefrac}
\usepackage{graphicx}
\shortcites{Ward2021} 


% \bibpunct{(}{)}{;}{a}{}{,}
\newcommand{\mco}[1]{\multicolumn{1}{c}{#1}}
\newcommand{\mct}[1]{\multicolumn{2}{c}{#1}}
\newcommand{\X}{$\times$ }
\newcommand{\hs}{\hspace{15pt}}
\newcommand{\Cov}{\operatorname{Cov}}

% Attempt to squeeze more floats in
\renewcommand\floatpagefraction{.9}
\renewcommand\topfraction{.9}
\renewcommand\bottomfraction{.9}
\renewcommand\textfraction{.1}
\setcounter{totalnumber}{50}
\setcounter{topnumber}{50}
\setcounter{bottomnumber}{50}


% Redefine the \appendix command
\usepackage{titlesec}
\usepackage{chngcntr} % Package to change the counter

% Define a new counter for the appendix sections
\newcounter{appendixsection}
\newcommand{\appsection}[1]{
    \stepcounter{appendixsection}
    \section*{Supplementary Online Appendix S\arabic{appendixsection}: #1}
    \addcontentsline{toc}{section}{Supplementary Online Appendix S\arabic{appendixsection}: #1}
    \setcounter{figure}{0} % Reset figure counter for each appendix section
    \setcounter{table}{0} % Reset table counter for each appendix section
    \renewcommand{\thefigure}{S\arabic{appendixsection}.\arabic{figure}} % Custom figure numbering
    \renewcommand{\thetable}{S\arabic{appendixsection}.\arabic{table}} % Custom table numbering
    \renewcommand{\theHfigure}{\thefigure} % Ensure unique hyperlink targets for figures
    \renewcommand{\theHtable}{\thetable} % Ensure unique hyperlink targets for tables
}


\title{
Supplementary Online Appendix to 

\bigskip

Impact of Twin Lockdowns on Hunger, Labor Market Outcomes, and
Household Coping Mechanisms: Evidence from Uganda
}

\author{Claus C. Pörtner \and Shamma A. Alam \and Ishraq Ahmed}

\date{August 2024}

\begin{document}


\maketitle

\appendix

\appsection{Alternative Google Mobility Measures}\label{alternative-google-mobility-measures}

\begin{figure}[htp]
\centering
\includegraphics[width=\linewidth, keepaspectratio]{../figures/mobility_national_workplaces.pdf}
\caption{Percentage changes in the number of visitors to workplace
locations}\label{fig:mobility_national_workplaces}
\end{figure}

\begin{figure}
\centering
\includegraphics[width=\linewidth, keepaspectratio]{../figures/mobility_national_transit.pdf}
\caption{Percentage changes in the number of visitors to transit
locations}\label{fig:mobility_national_transit}
\end{figure}

\begin{figure}
\centering
\includegraphics[width=\linewidth, keepaspectratio]{../figures/mobility_national_grocery.pdf}
\caption{Percentage changes in the number of visitors to grocery and
pharmacy locations}\label{fig:mobility_national_grocery}
\end{figure}

\begin{figure}
\centering
\includegraphics[width=\linewidth, keepaspectratio]{../figures/mobility_national_parks.pdf}
\caption{Percentage changes in the number of visitors to
park}\label{fig:mobility_national_parks}
\end{figure}

\begin{figure}
\centering
\includegraphics[width=\linewidth, keepaspectratio]{../figures/mobility_national_residential.pdf}
\caption{Percentage changes in the time spent at residential
locations}\label{fig:mobility_national_residential}
\end{figure}

\clearpage


\appsection{Seasonality}\label{seasonality}

\begin{figure}[htp]
\centering
\includegraphics[width=\linewidth, keepaspectratio]{../figures/seasonality_comparison.pdf}
\caption{Results with restricted sample to examine
seasonality}\label{fig:seasonality_comparison}
\end{figure}

\begin{figure}
\centering
\includegraphics[width=\linewidth, keepaspectratio]{../figures/seasonality_urban.pdf}
\caption{Results for urban only to examine
seasonality}\label{fig:seasonality_urban}
\end{figure}

\clearpage

\appsection{Regional}\label{regional}

\begin{figure}[htp]
\centering
\includegraphics[width=\linewidth, keepaspectratio]{../figures/food_insecurity_by_region_survey_round_3_levels.pdf}
\caption{Food insecurity by region and survey round of the Uganda
high-frequency phone survey on COVID-19 together with severe lockdown
periods in grey}\label{fig:region_descriptive}
\end{figure}

\begin{figure}
\centering
\includegraphics[width=\linewidth, keepaspectratio]{../figures/mobility_regional_workplaces.pdf}
\caption{Regional distribution of percentage changes in the number of
visitors to workplace locations}\label{fig:mobility_regional_workplaces}
\end{figure}

\begin{figure}
\centering
\includegraphics[width=\linewidth, keepaspectratio]{../figures/mobility_regional_transit.pdf}
\caption{Regional distribution of percentage changes in the number of
visitors to transit locations}\label{fig:mobility_regional_transit}
\end{figure}

\begin{figure}
\centering
\includegraphics[width=\linewidth, keepaspectratio]{../figures/mobility_regional_grocery.pdf}
\caption{Regional distribution of percentage changes in the number of
visitors to grocery and pharmacy
locations}\label{fig:mobility_regional_grocery}
\end{figure}

\begin{figure}
\centering
\includegraphics[width=\linewidth, keepaspectratio]{../figures/mobility_regional_parks.pdf}
\caption{Regional distribution of percentage changes in the number of
visitors to parks}\label{fig:mobility_regional_parks}
\end{figure}

\begin{figure}
\centering
\includegraphics[width=\linewidth, keepaspectratio]{../figures/mobility_regional_residential.pdf}
\caption{Regional distribution of percentage changes in the time spent
at residential locations}\label{fig:mobility_regional_residential}
\end{figure}

\clearpage

\appsection{Labor Market Outcomes}\label{labor-market-outcomes}

\begin{figure}[htp]
\centering
\includegraphics[width=\linewidth, keepaspectratio]{../figures/transition_absolute.pdf}
\caption{Sectoral distribution by survey round in
percent}\label{fig:transition_absolute}
\end{figure}

\clearpage

\appsection{Change in Agricultural Strategy}\label{change-in-agricultural-strategy}

In this section, we examine whether agricultural households change their
agricultural strategy to better cope with the effects of the lockdowns.
The survey asked in Rounds 1, 4, and 7 to households engaged in planting
activities whether they changed their ``planting activities in the
current agricultural season because of changes in the country or
community due to coronavirus?''. During the first lockdown, 23.2 percent
of agricultural households reported changing their planting activities
because of the pandemic, as did 18.5 percent during the second lockdown.
This is as opposed to 5 percent for the non-lockdown period of Round 4.

We create an indicator variable where 1 represents a change in planting
activities, and 0 represents no change. We present the estimates of the
impact of lockdowns on changes in planting activities in
Figure \ref{fig:ag_plant_change}. The estimates show that the first
lockdown led to a 57 percentage points increase in the likelihood of
changing crop planting activities, and the second lockdown led to a 26
percentage points increase compared to Round 4.

For households with a change in activities, the survey also asked them
how they changed their activities. This allows us to shed more light on
how agricultural households attempted to change their farming strategy
to cope with the effect of the shock. Table \ref{tab:ag_changes} shows
that the biggest change was a change in the use of farm areas, where 8.7
percent reported a reduction, and 9 percent reported an increase in the
use of farm areas after the first lockdown. It is followed by changes in
the number of varieties of crops produced, where both an increase
(4.1\%) and a decrease (2.6\%) in variety are mentioned after the first
lockdown. Only a small fraction of farmers delayed planting (1.3\%) or
abandoned crop farming (1.5\%) altogether for that season after the
first lockdown.

\begin{figure}[htp]
\centering
\includegraphics[width=\linewidth, keepaspectratio]{../figures/ag_plant_change.pdf}
\caption{Estimated change in planting activities, relative to Round
4}\label{fig:ag_plant_change}
\end{figure}

\begin{table}[hbtp!]
\begin{center}
\begin{small}
\begin{threeparttable}
\caption{Average changes in                               
 agricultural strategy                                      
 because of COVID-19 (in                                    
 percentages)}
\label{tab:ag_changes}
\begin{tabular}{@{} l  D{.}{.}{2.2} D{.}{.}{2.2}  D{.}{.}{2.2}  @{}}
\toprule 
        & \multicolumn{3}{c}{Survey Round} \\ \cmidrule(lr){2-4} 
        & \multicolumn{1}{c}{1} & \multicolumn{1}{c}{4}  & \multicolumn{1}{c}{7} \\ 
\midrule 
Change planting because of COVID-19 & 23.2\% & 5.0\% & 18.5\% \\ 
 Reduced farm area & 8.7\% & 2.1\% & 10.0\% \\ 
 Increased farm area & 9.0\% & 0.0\% & 3.8\% \\ 
 Planted less variety/number of crops & 4.1\% & 1.4\% & 6.2\% \\ 
 Planted more variety/number of crops & 2.6\% & 1.1\% & 1.6\% \\ 
 Delayed planting & 1.3\% & 0.3\% & 2.8\% \\ 
 Planted crops that mature quickly & 1.0\% & 0.0\% & 1.6\% \\ 
 Abandoned crop farming & 1.5\% & 0.0\% & 0.5\% \\ 
 \bottomrule
\end{tabular}
\begin{tablenotes}
\item \hspace*{-0.5em} \scriptsize \emph{Source:} Authors' analysis based on data from the Uganda High-Frequency Phone Survey, Rounds 1, 4, and 7. 
\item \hspace*{-0.5em} \scriptsize \textit{Note:} Questions on crop planting activities are only asked in rounds 1, 4, and 7. \end{tablenotes}
\end{threeparttable}
\end{small}
\end{center}
\end{table}


\end{document}